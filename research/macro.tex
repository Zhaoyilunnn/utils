%% comments
\newcommand{\ylz}[1]{\textcolor{purple}{#1}}
\newcommand{\fixme}[1]{[\textcolor{red}{FIXME:#1}]}
\newcommand{\tbd}[1]{[\textcolor{red}{TODO: #1}]}
\newcommand{\xf}[1]{{\color{blue}#1}}
\newcommand{\zkd}[1]{{\color{purple}#1}}
%% comments


% fonts
\newcommand{\tb}[1]{\textbf{#1}} % bold font
\newcommand{\iname}[1]{\texttt{\textcolor{blue}{#1}}} % for instruction name
\newcommand{\code}[1]{\texttt{#1}}
% fonts


% revision highlight
\newcommand{\rev}[1]{{\color{purple}#1}}
% \renewcommand{\rev}[1]{#1} % uncomment to remove highlight color

% D-HISQ
% DIRAC: DIstRibuted quAntum Control
% \newcommand{\name}{\tbd{\textsc{QuDist}}}
\newcommand{\name}{\textsc{Distributed-HISQ}\xspace}
\newcommand{\sync}{\textsc{BISP}\xspace}
\newcommand{\fmc}{\texttt{recv.c}\xspace}
\newcommand{\fmq}{\texttt{recv.q}\xspace}
\newcommand{\fmt}{\texttt{FMT}\xspace}
\newcommand{\mec}{\texttt{send.c}\xspace}
\newcommand{\meq}{\texttt{send.q}\xspace}
\newcommand{\met}{\texttt{MET}\xspace}
\newcommand{\waittall}{\texttt{WAITT\_ALL}\xspace}
\newcommand{\waitt}{\texttt{WAITT}\xspace}
\newcommand{\waita}{\texttt{WAITA}\xspace}
\newcommand{\waiti}{\texttt{WAITI}\xspace}
\newcommand{\waitr}{\texttt{WAITR}\xspace}
\newcommand{\router}{router\xspace}



\newcommand{\squishlist}{
   \begin{list}{$\bullet$}
    {
    \setlength{\itemsep}{0pt}      \setlength{\parsep}{0pt}
      \setlength{\topsep}{3pt}       \setlength{\partopsep}{0pt}
      \setlength{\listparindent}{-2pt}
      \setlength{\itemindent}{-5pt}
      \setlength{\leftmargin}{1em} \setlength{\labelwidth}{0em}
      \setlength{\labelsep}{0.5em} } }

\newcommand{\squishend}{
    \end{list}  }


% level1-enum
\newcounter{outercounter}
\newcommand{\squishenum}{%
  \begin{list}{\arabic{outercounter}}{%
    \usecounter{outercounter}%
    \setlength{\itemsep}{0pt}%
    \setlength{\parsep}{0pt}%
    \setlength{\topsep}{3pt}%
    \setlength{\partopsep}{0pt}%
    \setlength{\listparindent}{-10pt}%
    \setlength{\itemindent}{-2pt}%
    \setlength{\leftmargin}{1em}%
    \setlength{\labelwidth}{0em}%
    \setlength{\labelsep}{0.5em}%
    \let\makelabel=\makeplainlabel%
  }%
}
\newcommand{\makeplainlabel}[1]{#1.\hfill}
\newcommand{\squishenumend}{\end{list}}

% level2-enum
\newcounter{innercounter}
\newcommand{\squishenumII}{%
  \begin{list}{\alph{innercounter}}{%
    \usecounter{innercounter}%
    \setlength{\itemsep}{0pt}%
    \setlength{\parsep}{0pt}%
    \setlength{\topsep}{3pt}%
    \setlength{\partopsep}{0pt}%
    \setlength{\listparindent}{-2pt}%
    \setlength{\itemindent}{-10pt}%
    \setlength{\leftmargin}{1em}%
    \setlength{\labelwidth}{0em}%
    \setlength{\labelsep}{0.5em}%
    \let\makelabel=\makenestedlabel%
  }%
}
\newcommand{\makenestedlabel}[1]{(#1)\hfill}
\newcommand{\squishenumIIend}{\end{list}}
\newcommand*\circled[1]{\tikz[baseline=(char.base)]{
  \node[shape=circle,draw,fill=black,text=white,font=\bf,inner sep=0.5pt] (char)
  {\scriptsize#1};
}}

\newcommand*\circledwhite[1]{\tikz[baseline=(char.base)]{
  \node[shape=circle,draw,fill=white,text=black,font=\bf,inner sep=0.5pt] (char)
  {\scriptsize#1};
}}

\newcommand{\etal}{et al.}
\newcommand{\ie}{i.e.}
\newcommand{\eg}{e.g.}
\newcommand{\etc}{etc.}

\newcommand{\putsec}[2]{\vspace{-0.1in}\section{#2}\label{sec:#1}\vspace{-0.05in}}
\newcommand{\putssec}[2]{\vspace{-0.0in}\subsection{#2}\label{ssec:#1}\vspace{-0.0in}}
\newcommand{\putsssec}[2]{\vspace{-0.0in}\subsubsection{#2}\label{sssec:#1}\vspace{-0.0in}}
%\newcommand{\putsssecX}[1]{\vspace{0.0in}\subsubsection*{#1}\vspace{0.0in}}
\newcommand{\putsssecX}[1]{\vspace{0.0in}\noindent\textbf{#1:}}

%\newcommand{\figref}[1]{Figure~\ref{fig:#1}}
\newcommand{\figref}[1]{Fig.~\ref{fig:#1}}
%\newcommand{\eqnref}[1]{Equation~\ref{eq:#1}}
\newcommand{\eqnref}[1]{Eq.~\ref{eq:#1}}
\newcommand{\tabref}[1]{Table~\ref{tab:#1}}
\newcommand{\secref}[1]{Section~\ref{sec:#1}}
\newcommand{\ssecref}[1]{Section~\ref{ssec:#1}}
\newcommand{\sssecref}[1]{Section~\ref{sssec:#1}}


%% Comments macros %%%%%%%%%%%%%%%%%

\newcommand{\COMM}[1]{#1}
\newcommand{\yilun}[1]{\COMM{\textcolor{cyan}{\sf\bfseries Yilun: #1}}}
\newcommand{\blc}[1]{\COMM{\textcolor{blue}{\sf\bfseries[BL: #1]}}}
% \newcommand{\blc}[1]{\textcolor{blue}{[BL: #1]}}
% \newcommand{\xulong}[1]{\COMMENT{\textcolor{brown}{\sf\bfseries Xulong: #1}}}
% \newcommand{\yanan}[1]{\COMMENT{\textcolor{green}{\sf\bfseries Yanan: #1}}}
% \newcommand{\amanda}[1]{\COMMENT{\textcolor{purple}{\sf\bfseries Amanda: #1}}}
% \newcommand{\devin}[1]{\COMMENT{\textcolor{orange}{\sf\bfseries Devin: #1}}}

% \definecolor{shiv_purple}{rgb}{0.6       ,  0.19607843,  0.8}
% \newcommand{\shiv}[1]{\COMMENT{\textcolor{shiv_purple}{\sf\bfseries Shiv: #1}}}
% \newcommand{\todo}[1]{\COMMENT{{\color{red}\sf\bfseries [#1]}}}

% Uncomment this to remove the inline comments
% \renewcommand{\COMM}[1]{}


\newcommand{\TODO}[1]{#1}
\newcommand{\todo}[1]{\TODO{{\color{red}\sf\bfseries [#1]}}}
% \renewcommand{\TODO}[1]{}

\newcommand{\cnot}{$\mathrm{CNOT}$\xspace}
\newcommand{\cz}{$\mathrm{CZ}$\xspace}
\newcommand{\hgate}{$H$\xspace}
\newcommand{\xgate}{$X$\xspace}

